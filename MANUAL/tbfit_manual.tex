\documentclass[a4paper,12pt]{scrartcl}
\usepackage[colorlinks=True,linkcolor=blue,urlcolor=orange]{hyperref}
\usepackage{fancyvrb}
\usepackage{xcolor}
\usepackage{amssymb}

\makeatletter
\def\namedlabel#1#2{\begingroup
    #2%
    \def\@currentlabel{#2}%
    \phantomsection\label{#1}\endgroup
}
\def\verbatim@font{\linespread{1}\normalfont\ttfamily}
\makeatother

% Title Page
\title{A short guide to the Tight-Binding FITting (\tbfitname{}) package}
\author{Hyun-Jung Kim [Infant@kias.re.kr]}
\newcommand{\testsuite}{\textcolor{blue}{\texttt{testsuite}}}
\newcommand{\testcase}{\textcolor{green}{\texttt{testcase}}}
\newcommand{\testparameter}{\textcolor{red}{\texttt{testparameter}}}
\newcommand{\tbfitname}{\textcolor{blue!85!white}{\texttt{TBFIT}}}
\newcommand{\textred}[1]{\textcolor{red!85!white}{\texttt{#1}}}
\newcommand{\textblue}[1]{\textcolor{blue!85!white}{\texttt{#1}}}
\newcommand{\textgreen}[1]{\textcolor{green!50!black}{\texttt{#1}}}
\newcommand{\textpink}[1]{\textcolor{red!60!yellow}{\texttt{#1}}}
\begin{document}
\maketitle

\begin{abstract}
This document is to provide explanation for the input file arguments of the \tbfitname{} package.

\end{abstract}
\section*{System Requirements}
All the program has been written by modern \texttt{Fortran90} language. \texttt{LAPACK} library should be properly linked in the makefile.

\newpage
\part{User's Guide}

\section{INPUT tags of the \texttt{INCAR-TB}}

\begin{description}

    \item[\namedlabel{tag:GETBAND}{GET\_BAND}] $logical$ Default: \texttt{.TRUE.}
		If \texttt{.TRUE.} \tbfitname{} will perform tight-binding calculations 
		for band structure evaluation.
		
    \item[\namedlabel{tag:TBFIT}{TBFIT}] $logical$ Default: \texttt{.FALSE.}  
		\subitem \texttt{.TRUE.}  : Perform tight-binding parameter fitting 
			which is defined in \ref{tag:PFILE}. After fitting is 
			completed, whatever it is converged or not, 
			additional tight binding calculations as defined 
			in the \texttt{INCAR-TB} will be performed.
		\subitem \texttt{.FALSE.} : Do not perform fitting procedures. 
			In this case, regular tight binding calculations 
        	will be performed.

    \item[\namedlabel{tag:MITER}{MITER}] $integer$ Default: \texttt{100} \\
		Maximum number of iteration for the fitting procedures

    \item[\namedlabel{tag:LSTYPE}{LSTYPE}] $integer$ Default: \texttt{LMDIF} \\
        Method for parameter fitting.
		Currently, \tbfitname{} only supports \texttt{LMDIF} method which is the 
		Levenberg-Marquardt method\footnote{ Kenneth Levenberg, 
		"A Method for the Solution of Certain Non-Linear Problems in Least Squares" 
		$Quarterly$ $of$ $Applied$ $Mathematics$ 2, 164 (1944).}$^,$
		\footnote{Donald Marquardt, "An Algorithm for Least-Squares Estimation of 
		Nonlinear Parameters" $SIAM$ $Journal$ $on$ $Applied$ $Mathematics$ 11, 
		431 (1963).} using finite-difference for Jacobian.

    \item[PTOL \& FTOL] $real$ Default: 0.00001\\
      	Tolerence of iteration of the fitting procedures. 
	  	\texttt{FTOL} is a tolerence for the difference between target and 
      	calculated data from tight binding method. 
	  	\texttt{PTOL} is as tolerence for the tight binding parameters.
      	Normally, both values below 0.00001 is sufficient to reach 
	  	a local minima.
    \item[K\_UNIT] $string$ Default: \texttt{ANGSTROM}
		\subitem \texttt{ANGSTROM}   : the unit of the $k$-point will be written
			in \AA$^{-1}$ unit.
		\subitem \texttt{RECIPROCAL} : the unit of the $k$-point will be written
			in reciprocal unit (fractional).
    \item [\namedlabel{tag:PFILE}{PFILE}] $string$
		File name for tight-binding parameters. Default: \texttt{PARAM\_FIT.dat}
		For the details, see Sec.\ref{tag:PFILE-detail}.
    \item[POFILE] $string$
		Output file name for tight-binding parameters written after fitting 
		procedures. Default: \texttt{PARAM\_FIT.new.dat}
    \item[\namedlabel{tag:IS_SK}{IS\_SK} or SLATER\_KOSTER] $logical$ 
		\subitem \texttt{.TRUE.}  : Slater-Koster type of hopping parameters will 
			be assumed.
		\subitem \texttt{.FALSE.}  : User defined or direct hopping parameters will 
			be assumed.
    \item[\namedlabel{tag:EFILE}{EFILE}] $string, integer$ \\
		File name for the $target$ band structure for the fitting procedures. 
		If the second $integer$ $n$ is followed by,
		\tbfitname{} will read $n$-th column as a target band. Default is $n$=2.
		\begin{verbatim}
		 EFILE DFT_BANDSTRUCTURE.dat 2
		\end{verbatim}

		\begin{Verbatim}[commandchars=\\\{\},gobble=4, frame=single, framesep=2mm, 
		   	label= EFILE DFT\_BANDSTRUCTURE.out example ,
		   	labelposition=bottomline]
        \# 1st eigen value
        \# k-path  energy(eV)
          0.00000  -12.36137  
          0.01693  -12.36162  
          0.03386  -12.36118  
            [...]
          0.16932  -12.33324  
          0.18625  -12.32696  
          0.20319  -12.32014  


        \# 2nd eigen value
        \# k-path  energy(eV)
          0.00000  -12.36137  
          0.01693  -12.36041  
          0.03386  -12.35875  
             [...]
          0.16932  -12.32136  
          0.18625  -12.31394  
          0.20319  -12.30600  


            [...]

		\end{Verbatim}
    \item[\namedlabel{tag:GFILE}{GFILE}] $string$ Default: \texttt{POSCAR-TB}\\
        File name for the geometry and atomic orbital informations. 
		The format is exactly same as \texttt{POSCAR} of \href{https://www.vasp.at}{\texttt{VASP}}
		program. For the details of setting atomic orbitals, see Sec.\ref{tag:GFILE-detail}.

        \begin{Verbatim}[commandchars=\\\{\},gobble=4, frame=single, framesep=2mm, 
            label= POSCAR-TB example: MoS$_2$ with \texttt{Mo}-$d$ and \texttt{S}-$s$$p$,
            labelposition=bottomline]
     MoS2 \# comment
       1.00000000000000  \# scaling factor
         3.1716343    0.000000    0.00000  \# lattice vector a1
         1.5858171    2.746715    0.00000  \# lattice vector a2
         0.0000000    0.000000   15.00000  \# lattice vector a3
       Mo  S                               \# atomic species
         1   2                          \# number of atoms per species
     Direct     \# coordinate type (direct or cartesian)
      0.00000 0.00000 0.50000 dz2 dxy dx2 dyz dxz  # coord, orbital
      0.33333 0.33333 0.60645 s px py pz              
      0.33333 0.33333 0.39354 s px py pz

        \end{Verbatim}

    \item[\namedlabel{tag:KFILE}{KFILE}] $string$ Default: \texttt{KPOINTS\_BAND}\\
        File name for the $k$-point setting.
        The format is exactly same as \texttt{KPOINTS} of \href{https://www.vasp.at}{\texttt{VASP}}
        program.

        \begin{Verbatim}[commandchars=\\\{\},gobble=4, frame=single, framesep=2mm, 
            label= KPOINTS\_BAND $line$ $mode$ example, 
            labelposition=bottomline]
     k-points line mode example
       40  ! intersections
     Line-mode
     Reciprocal
       0.50000000  0.5000000 0 M
       0.33333333  0.6666666 0 K
       
       0.33333333  0.6666666 0 K
       0.00000000  0.0000000 0 G
       
       0.00000000  0.0000000 0 G
       0.66666666  0.3333333 0 K'

        \end{Verbatim}

        \begin{Verbatim}[commandchars=\\\{\},gobble=4, frame=single, framesep=2mm, 
            label= KPOINTS\_BAND $grid$ $mode$ example, 
            labelposition=bottomline]
     k-points grid mode example
       0 
     GMonkhorst-Pack #'G'amma centered grid mode
       4  4  1   \# grid  nk_1 nk_2 nk_3
       0  0  0   \# shift
        \end{Verbatim}

    \item[\namedlabel{tag:LOCCHG}{LOCCHG}] $logical$ Default: \texttt{.FALSE.}\\
        Setting tag for local potential.
        If \texttt{.TRUE.}, one should give proper local potential parameter 
		in your \ref{tag:PFILE} and should properly setup \ref{tag:param-locpot} 
		tag in your \ref{tag:GFILE}. For the details, see the explanation of
		\ref{tag:param-locpot} in Sec.\ref{tag:PFILE-detail}.

    \item[\namedlabel{tag:TYPMAG}{TYPMAG}] $string$ Default: \texttt{NONMAG}\\
        Setting tag for magnetic moment: \texttt{nonmagnetic}, \texttt{collinear}, 
		\texttt{noncollinear}
		If \texttt{collinear} and \texttt{noncollinear} tag is applied, 
		\ref{tag:MOMENT} or \ref{tag:MOMENT.C} in the \ref{tag:GFILE}
		should be set up appropriately.
		For details, see \ref{tag:MOMENT} of the Sec.\ref{tag:GFILE-detail}.

    \item[\namedlabel{tag:LSORB}{LSORB}] $logical$ Default: \texttt{.FALSE.}\\
        Setting tag for spin-orbit coupling. 
		If \texttt{.TRUE.}, $lambda\_orb\_spec$ should be properly defined in the
		\ref{tag:PFILE}. For details, see Sec.\ref{tag:PFILE-detail}

    \item[\namedlabel{tag:LORBIT}{LORBIT}] $logical$ Default: \texttt{.TRUE.}\\
        Setting tag for orbital decomposed output.
		If \texttt{.TRUE.} the absolute value of wavefunction coefficient 
		($<\psi_{nk}|$$\phi$$|\psi_{nk}>$) will be 
		printed out in \texttt{bandstructure\_TBA.dat} file.

    \item[\namedlabel{tag:IBAND}{IBAND}] $integer$ Default: \texttt{1}\\
		\texttt{IBAND} is the first eigenstate of the target data of 
		\ref{tag:EFILE}. This value will be used in the \ref{tag:WEIGHT}
		\texttt{SET} section.

    \item[\namedlabel{tag:FBAND}{FBAND}] $integer$ Default: \texttt{NEIG} \\
		\texttt{NEIG} : number of orbital basis of the system.
		\texttt{FBAND} is the last eigenstate of the target data of 
		\ref{tag:EFILE}. This value will be used in the \ref{tag:WEIGHT}
		\texttt{SET} section.

    \item[\namedlabel{tag:SCISSOR}{SCISSOR}] $integer, real$ \\
		If set, in the fitting procedures, target energy \texttt{EDFT($n$,$k$)}
		will be shift by amound of the scissor operation. 
		This operation works as follows: E$_{target}'(n,k)$ = 
		E$_{target}$($n$,$k$) + $e_{scissor}$ if $n$ $>$= i$_{scissor}$.
		Note that this operation is only valied if \ref{tag:TBFIT} is \texttt{.TRUE.}. 
    \begin{verbatim}
     SCISSOR 29 0.2  # i_scissor = 29 and e_scissor = 0.2 (eV)
    \end{verbatim}

    \item[\namedlabel{tag:ERANGE}{ERANGE}] $integer$ Default: \texttt{1 NEIG} \\
		If provided, the energy level between these energy window will be printed
		out in the \texttt{bandstructure\_TBA.dat} file. 
    \begin{verbatim}
     ERANGE  4400 4700  
    \end{verbatim}
		Above example means that the energy level from 4400$^{th}$ to 
		4700$^{th}$ will be printed. 
		This is particularly useful if you calculate very large
		systems. By setting \texttt{ERANGE} tag, you can save disk space
		a lot if \ref{tag:LORBIT} tag is turned on where orbital component
		information takes huge memory for larger systems.

    \item[\namedlabel{tag:SET}{SET}] $string$ \\
        Setting tag for various post processings, parameter constraints,
		and nearest neighbor setups, etc.
		Available list for the \texttt{SET} tags are as follows,
 	\subitem \ref{tag:CONSTRAINT}
 	\subitem \ref{tag:NNCLASS}
 	\subitem \ref{tag:RIBBON}
 	\subitem \ref{tag:BERRYC}
 	\subitem \ref{tag:ZAKPHASE}
 	\subitem \ref{tag:WCC}
 	\subitem \ref{tag:Z2}
 	\subitem \ref{tag:EFIELD}
 	\subitem \ref{tag:WEIGHT}
 	\subitem \ref{tag:DOS}
 	\subitem \ref{tag:EIGPLOT}
 	\subitem \ref{tag:STMPLOT}

\end{description}


\newpage

\section{Details of the \texttt{SET}}\label{tag:SET-detail}
Each \texttt{SET} tag should be ended up by \texttt{END} tag.

\begin{description}

    \item[\namedlabel{tag:STMPLOT}{STMPLOT}]
        Setting of integrated eigen state wavefunction $\Sigma |\psi_{nk}(r)|^{2}$ plot.
		Here, the summation runs over the eigen states within the energy window specified
		by \texttt{STM\_ERANGE} or equivalently \texttt{STM\_WINDOW}.

 \begin{Verbatim}[commandchars=\\\{\},gobble=4, frame=single, framesep=2mm, 
    label= STMPLOT setup example,
    labelposition=bottomline]
    \color{blue}SET \color{red}STMPLOT
      \textgreen{NGRID} 40 40 80 # GRID for CHGCAR-STM output (default = 0.1 ang).
      \textgreen{STM\_ERANGE} -1.0:0.0  # energy window
      \textgreen{RCUT} 6.0 # cut off radius(\AA). Beyond this will not be calculated.
      \textgreen{REPEAT_CELL} T T T  # repeat orbital for each lattice vector?
         # this logical tag is especially useful if you only 
         #consider center region of the very large cell.
         # If set "T T F", orbital contribution which is periodically 
         # repeated in a3 direction wll not be considered to calculate. 
         # Try this option if  you have very large cell and you are 
         # especially interested unitcell ceter.
    \color{blue}END \color{red}STMPLOT
 \end{Verbatim}

    \item[\namedlabel{tag:EIGPLOT}{EIGPLOT}]
        Setting of eigen state wavefunction $\psi_{nk}(r)$ or 
		charge density $|\psi_{nk}(r)|^{2}$ plot.

 \begin{Verbatim}[commandchars=\\\{\},gobble=4, frame=single, framesep=2mm, 
    label= EIGPLOT setup example,
    labelposition=bottomline]
    \color{blue}SET \color{red}EIGPLOT
      \textgreen{IEIG} 3 5   # index(es) n of eigen state.
      \textgreen{IKPT} 1 10  # index(es) k of k-point.
      \textgreen{NGRID} 40 40 80 # GRID for CHGCAR output (default = 0.1 ang).
      \textgreen{RORIGIN} 0.0 0.0 0.0 # shift of the origin of the cube file.
      \textgreen{WAVEPLOT} .TRUE. # plot wavefunction (.true.) or charge density.
      \textgreen{RCUT} 6.0 # cut off radius(\AA). Beyond this will not be calculated.
    \color{blue}END \color{red}EIGPLOT
 \end{Verbatim}


    \item[\namedlabel{tag:DOS}{DOS}]
        Setting of \texttt{Density of states (DOS)}.

 \begin{Verbatim}[commandchars=\\\{\},gobble=4, frame=single, framesep=2mm, 
    label= DOS setup example,
    labelposition=bottomline]
    \color{blue}SET \color{red}DOS
      \textgreen{GKGRID} 100 100 1   # set Gamma centered Monkhorst-Pack grid 
      \textgreen{KSHIFT} 0.0 0.0 0.0 # shift of k-grid (k-offset)
      \textgreen{PRINT_KPTS} .TRUE. IBZKPT-DOS_TB # print k-point to the file
      \textgreen{PRINT_EIG} .TRUE. 1:2 3 # print specified energy surface 
      \textgreen{PRINT_UNIT} RECIPROCAL # k-point unit (or ANGSTROM 1/A)
      \textgreen{SMEARING} 0.03 # gaussian smearing. Default = 0.025
      \textgreen{NEDOS} 2000    # number of grid points in energy window (erange)
      \textgreen{DOS_ERANGE} -20.0:10.0 # energy window to be plotted
      \textgreen{DOS_NERANGE} 1:NEIG # energy window to be calculated (integer)
      \textgreen{DOS_FNAME} DOS_TB_projected.dat # output file name for DOS output
    \color{blue}END \color{red}DOS
 \end{Verbatim}


    \item[\namedlabel{tag:EFIELD}{EFIELD}]
        Setting of \texttt{E-field}.

 \begin{Verbatim}[commandchars=\\\{\},gobble=4, frame=single, framesep=2mm, 
    label= EFIELD setup example,
    labelposition=bottomline]
    \color{blue}SET \color{red}EFIELD
      \textgreen{EFIELD}  0.0 0.0 0.1  # Efield along z direction
      \textgreen{EF\_ORIGIN}  0.0 0.0 0.345690593  # (in fractional coordinate)
     #\textgreen{EF\_CORIGIN} 0 0 0  # (in cartesian coordinate)
    \color{blue}END \color{red}EFIELD
 \end{Verbatim}


    \item[\namedlabel{tag:WEIGHT}{WEIGHT}]
		Setting of weight factor for the fitting procedures.
		\subitem KRANGE $integer$ : range of k-point where the weight factor is applied
		\subitem TBABND $integer$ : range of eigen states of the tight binding calculation 
		\subitem DFTBND $integer$ : range of eigen states of the target energy bands
		\subitem WEIGHT $real$ : weighting factor 
		\subitem ORBT\_I $ineteger$ : orbital index. $n^{th}$ orbital states will get a penalty
		\subitem SITE\_I $ineteger$ : site index. \texttt{ORBT\_I}$^{th}$  orbital state 
		at \texttt{SITE\_I} atom will get a penalty. This prohibit certain orbital character 
		to be stabilized from the fitting procedures.
		


 \begin{Verbatim}[commandchars=\\\{\},gobble=4, frame=single, framesep=2mm, 
    label= WEIGHT setup example,
    labelposition=bottomline]
    \color{blue}SET \color{red}WEIGHT
      \textgreen{KRANGE}  :            \textgreen{TBABND}  :    \textgreen{DFTBND} IBAND:FBAND \textgreen{WEIGHT} 1
      \textgreen{KRANGE}  :            \textgreen{TBABND} 17:20 \textgreen{DFTBND} 17:20       \textgreen{WEIGHT} 6 
      \textgreen{KRANGE} 20:60 100:140 \textgreen{TBABND} 17:20 \textgreen{DFTBND} 17:20       \textgreen{WEIGHT} 20
      \textgreen{KRANGE} 1     \textgreen{TBABND}  7    \textgreen{ORBT\_I} 1  \textgreen{SITE\_I} Mo1 \textgreen{PENALTY} 200
    \color{blue}END \color{red}WEIGHT
 \end{Verbatim}


 	\item[\namedlabel{tag:CONSTRAINT}{CONSTRAINT TBPARAM}]
		Setting for parameter constraints for the fitting and calculation.
		The value of the specified two parameter will be kept same during the
		fitting and tight-binding calculations.

 \begin{Verbatim}[commandchars=\\\{\},gobble=4, frame=single, framesep=2mm, 
    label= CONSTRAINT setup example,
    labelposition=bottomline]
    \color{blue}SET \color{red}CONSTRAINT
        \textgreen{e_py_S}  = \textgreen{e_px_S} 
    \color{blue}END \color{red}CONSTRAINT
 \end{Verbatim}

		If the second argument `=' is replaced by `==' and the third argument
		is not present, then this parameter will not be fitted and its initial
		guess as defined in \ref{tag:PFILE} will be fixed during the fitting procedures.
		Note that, exactly same effect can be achieved by putting `\texttt{FIXED}' tag
		at the parameter specification line of the \ref{tag:PFILE}, and the detailed 
		explanation can be found in \ref{tag:param-fix} of Sec.\ref{tag:PFILE-detail}.

 	\item[\namedlabel{tag:NNCLASS}{NN\_CLASS}]
		Setting for nearest neighbor set up. \\
		If the distance between two atomic species (For example, \texttt{Mo} 
		and \texttt{S}) are 1st nearest type, and its upper limit is 3.2 
		angstrom (e.g., below this value will be regarded as the pair), thene
		we can set as follows,
        \begin{verbatim}
         	Mo-S : 3.2  R0 3.171634
        \end{verbatim}  

		Here, number of dash '-' occurance between two atomic species indicates 
		the distance class $n$, and the above example represents 1st nearest 
		hopping between \texttt{Mo} and \texttt{S}. The following \texttt{R0}
		tag defines optimal bonding distance between two neighbor pair. 
		This value will be used in the scaling function for the distance
		dependent hopping parameter.

 \begin{Verbatim}[commandchars=\\\{\},gobble=4, frame=single, framesep=2mm, 
    label= NN\_CLASS setup example,
    labelposition=bottomline]
    \color{blue}SET \color{red}NN_CLASS
        \textgreen{Mo-Mo}  : 3.2   \textgreen{R0} 3.171634
        \textgreen{S-S}    : 3.28  \textgreen{R0} 3.171634  
        \textgreen{S--S}   : 3.2   \textgreen{R0} 3.193724  
        \textgreen{Mo-S}   : 2.5   \textgreen{R0} 2.429624  
    \color{blue}END \color{red}NN_CLASS
 \end{Verbatim}


 	\item[\namedlabel{tag:RIBBON}{RIBBON}]
		Setting for nanoribbon calculations. \\
		At the initial stages of the calculations, \tbfitname{} will generate
		\ref{tag:GFILE}\texttt{-ribbon} with the settings bellow.
		\subitem NSLAB $integer$ : multiplication of unitcell along each direction
		\subitem VACUUM  $real$  : vacuum spacing along each direction.
		\subitem KFILE\_R $real$  : \ref{tag:KFILE} for ribbon band structure. 
			Default: \ref{tag:KFILE}
		\subitem PRINT\_ONLY\_R $logical$  : if \texttt{.TRUE.} the geometry
		file will be generated with \texttt{-ribbon} suffix to the \ref{tag:GFILE}
		and the program will imedietly stops. Default: \texttt{.FALSE.}
		
 \begin{Verbatim}[commandchars=\\\{\},gobble=4, frame=single, framesep=2mm, 
    label= Ribbon calculation setup,
    labelposition=bottomline]
    \color{blue}SET \color{red}RIBBON
        \color{green!50!black}NSLAB        \color{black}1 20  1
        \color{green!50!black}VACUUM       \color{black}0 20  0
        \color{green!50!black}KFILE_R      \color{black}KPOINTS_RIBBON
        \color{green!50!black}PRINT_ONLY_R \color{black}.FALSE. or. TRUE.
    \color{blue}END \color{red}RIBBON
 \end{Verbatim}

%   SET  Z2 # setup for evaluating Z2 index
%        Z2_ERANGE    1:8   # up to occupied states!! (very important)
%        Z2_DIMENSION 2D:B3 # or 3D or 2D:b3, 1D:b1
%        Z2_NKDIV    11 111  # number of division of the k-path (odd number suggested)
%   END  Z2

    \item[\namedlabel{tag:Z2}{Z2\_INDEX}] Automatic calculations for topological index 
        $[\nu_0\; \nu_1, \nu_2, \nu_3]$ for 3D or $\mathbb{Z}_{2}$ for 2D via \ref{tag:WCC} method. 
  		(The current version does not support
        \texttt{MPI} parallelism for the Wannier charge center evaluation subroutine.
        Hence, do not use multicore for this particular calculation.
        Nevertheless, the compilation with \texttt{-DMPI} option is accepted.)
        The output will be written at \texttt{Z2.WCC.}\textgreen{plane\_index}\texttt{.dat} and
        \texttt{Z2.GAP.}\textgreen{plane\_index}\texttt{.dat}.
        Here, \textgreen{plane\_index} indicates one of six $B_i$-$B_j$ plane with $B_k$ = 0 or $\pi$.
        For example, if \textgreen{plane\_index} = \textgreen{0.0-B3.B1\_B2-PLANE}, then it contains
        WCC information of $B1$-$B2$ plane with $k_z$ = $\pi$.
        
 \begin{Verbatim}[commandchars=\\\{\},gobble=4, frame=single, framesep=2mm, 
    label= Z2 index calculation using WCC method,
    labelposition=bottomline]
    \color{blue}SET \color{red}Z2_INDEX
        \color{green!50!black}Z2_ERANGE    \color{black}1:28  # upto occupied
        \color{green!50!black}Z2_DIMENSION \color{black}3D  # or 2D:kz (2D WCC plane perpendicular to kz)
        \color{green!50!black}Z2_NKDIV  \color{black}21 21 # k-grid for KPATH and k-direction for WCC
        \color{green!50!black}Z2_CHERN  \color{black}.TRUE. # 1st Chern number of given bands with ERANGE
    \color{blue}END \color{red}Z2_INDEX
 \end{Verbatim}

    \item[\namedlabel{tag:WCC}{WCC}]Wannier Charge Center or Wilson loop calculation settings
        (The current version does not support
        \texttt{MPI} parallelism for the Wannier charge center evaluation subroutine. 
        Hence, do not use multicore for this particular calculation. 
        Nevertheless, the compilation with \texttt{-DMPI} option is accepted.)
 \begin{Verbatim}[commandchars=\\\{\},gobble=4, frame=single, framesep=2mm, 
    label= Wannier charge center (WCC) setup: kz 0.5 (shift),
    labelposition=bottomline]
    \color{blue}SET \color{red}WCC
        \color{green!50!black}WCC_ERANGE \color{black}1:28  # upto occupied
        \color{green!50!black}WCC_FNAME  \color{black}WCC.OUT.dat 
        \color{green!50!black}WCC_FNAME_GAP \color{black}WCC.GAP.dat  # largest gap will be written
        \color{green!50!black}WCC_KPATH  \color{black}0 0 0  1 0 0 # k_init -> k_end (ex, along b1)
        \color{green!50!black}WCC_KPATH_SHIFT  \color{black}0 0 0.5 # kpoint shift along b3 direction
        \color{green!50!black}WCC_DIREC  \color{black}2 #k-direction for WCC evolution (1:b1, 2:b2, 3:b3)
        \color{green!50!black}WCC_NKDIV  \color{black}21 21 # k-grid for KPATH and k-direction (odd number)
        \color{green!50!black}WCC_CHERN  \color{black}.TRUE. # 1st Chern number of given bands with ERANGE
    \color{blue}END \color{red}WCC
 \end{Verbatim}


 	\item[\namedlabel{tag:ZAKPHASE}{ZAK\_PHASE}]
		Setting for Zak phase calculations. (The current version does not support
        \texttt{MPI} parallelism for the Zak phase evaluation subroutine. Hence, do not use
        multicore for this particular calculation. However, the compilation with 
        \texttt{-DMPI} option is accepted.)

 \begin{Verbatim}[commandchars=\\\{\},gobble=4, frame=single, framesep=2mm, 
    label= Zak phase setup,
    labelposition=bottomline]
    \color{blue}SET \color{red}ZAK\_PHASE
        \color{green!50!black}ZAK_ERANGE \color{black}1:28  # upto occupied
        \color{green!50!black}ZAK_FNAME  \color{black}ZAK_PHASE.OUT.dat 
        \color{green!50!black}ZAK_KPATH  \color{black}0 0 0  1 0 0 # k_init -> k_end (ex, along b1)
        \color{green!50!black}ZAK_DIREC  \color{black}2 #k-direction for Zak phase evolution (1:b1, 2:b2, 3:b3)
        \color{green!50!black}ZAK_NKDIV  \color{black}21 21 # k-grid for KPATH and k-direction
    \color{blue}END \color{red}ZAK\_PHASE
 \end{Verbatim}

    \item[\namedlabel{tag:BERRYC}{BERRY\_CURVATURE}]
		Setting for Berry curvature calculations.
 \begin{Verbatim}[commandchars=\\\{\},gobble=4, frame=single, framesep=2mm, 
	label= Berrycurvature setup,
	labelposition=bottomline]
    \color{blue}SET \color{red}BERRY_CURVATURE
        \color{green!50!black}BERRYC_METHOD \color{black}KUBO # .or. RESTA(not yet supported)
        \color{green!50!black}BERRYC_ERANGE \color{black}17:18
        \color{green!50!black}BERRYC_FNAME  \color{black}BERRYCURV.17-18 # output will be \textgreen{BERRYC\_FNAME}.dat
        \color{green!50!black}BERRYC_DIMENSION \color{black}2D:B3  # 2D plane perpendicular to kz)
    \color{blue}END \color{red}BERRY_CURVATURE
 \end{Verbatim}


\end{description}


\section{Details of the format of \texttt{GFILE}}\label{tag:GFILE-detail}
%\subsubsection*{Input Syntax for Regular Expression Extraction Pattern}
\begin{description}

    \item[\namedlabel{tag:ATOMICORB}{Atomic orbital setup}] $string$ \\
        Hydrogen-like atomic orbital can be specified for the orbital basis.
        The possible orbital basises are\footnote{Please note that current 
		version does not support the $f$ orbitals.
        However, we will include $f$ in the future release of \tbfitname{}. 
		For the \texttt{Slater-Koster} tables of $f$ orbitals, please see
		[\texttt{K. Lendi, 
		\href{https://journals.aps.org/prb/abstract/10.1103/PhysRevB.9.2433}
		{Phys. Rev. B 9, 2433 (1974)}}].}:
        \begin{verbatim}
         s px py pz dz2 dxy dx2 dxz dx2
        \end{verbatim}


\begin{Verbatim}[commandchars=\\\{\},gobble=4, frame=single, framesep=2mm, 
    label= setup of atomic orbital basis in \ref{tag:GFILE},
    labelposition=bottomline]
  0.0 0.0 0.0 s px py pz  \# s, px, py, and pz orbitals at ATOM_A
  0.0 0.0 0.5 s px py pz  \# s, px, py, and pz orbitals at ATOM_B

\end{Verbatim}

    \item[\namedlabel{tag:CUSTOM}{Custumized atomic orbital setup}] $string$ \\
        If someone does not want to use \texttt{Slater-Koster} type interatomic
        hopping parameter, customized atomic orbital can be defined instead.
        In this case, distance and hopping pair dependent parameterization
        should be properly defined in the \ref{tag:PFILE}.

\begin{Verbatim}[commandchars=\\\{\},gobble=4, frame=single, framesep=2mm, 
    label= setup of custumized atomic orbital name $cp1$,
    labelposition=bottomline]
  0.0 0.0 0.0 cp1  \# cp1 orbital at ATOM_1
  0.0 0.0 0.5 cp1  \# cp1 orbital at ATOM_2

\end{Verbatim}


    \item[\namedlabel{tag:MOMENT}{MOMENT}] tag $real$ \\
		Magnetic moment for Each atomic orbital can be assigned as follows,
	\subitem collinear case: \texttt{0.0}
    \subitem noncollinear case: \texttt{0.0 0.0 0.0 [$M$ $\theta$ $\phi$]}\\
\begin{Verbatim}[commandchars=\\\{\},gobble=4, frame=single, framesep=2mm, 
    label= usage of $moment$ tag in \ref{tag:GFILE} with $collinear$ magnetism,
    labelposition=bottomline]
  0.0 0.0 0.0 px py pz \textgreen{moment} 0 0  1 \# spin-up for pz
  0.0 0.0 0.5 px py pz \textgreen{moment} 0 0 -1 \# spin-dn for pz

\end{Verbatim}

\begin{Verbatim}[commandchars=\\\{\},gobble=4, frame=single, framesep=2mm, 
    label= usage of $moment$ tag in \ref{tag:GFILE} with $noncollinear$ magnetism,
    labelposition=bottomline]
  0.0 0.0 0.0 px py pz \textgreen{moment} 0 0 0  0 0 0  1 0 0  \# spin-up for pz
  0.0 0.0 0.5 px py pz \textgreen{moment} 0 0 0  0 0 0 -1 0 0  \# spin-dn for pz

\end{Verbatim}

    \item[\namedlabel{tag:MOMENT.C}{MOMENT.C}] tag $real$ \\
        Similar to \ref{tag:MOMENT} but in \texttt{noncollinear} case,
		the 1$^{st}$, 2$^{nd}$, and 3$^{rd}$ value represents, $m_x$, $m_y$, and $m_z$, respectively.
		Here, $x$, $y$, and $z$ represents the cartesian axis.
    \subitem noncollinear case: \texttt{0.0 0.0 0.0 [$M_x$ $M_y$ $M_z$]}\\

\begin{Verbatim}[commandchars=\\\{\},gobble=4, frame=single, framesep=2mm, 
    label= usage of $moment.c$ tag in \ref{tag:GFILE} with $noncollinear$ magnetism,
    labelposition=bottomline]
  0.0 0.0 0.0 px py pz \textgreen{moment} 0 0 0  0 0 0  0 0  1 \# spin-up for pz
  0.0 0.0 0.5 px py pz \textgreen{moment} 0 0 0  0 0 0  0 0 -1 \# spin-dn for pz

\end{Verbatim}

\end{description}

\section{Details of the format of \texttt{PFILE}}\label{tag:PFILE-detail}
%\subsubsection*{Input Syntax for Regular Expression Extraction Pattern}
\begin{description}

    \item[\namedlabel{tag:param-onsite}{ONSITE parameters}] $real$ \\
		Onsite prameters for each atomic orbital should have the prefix
		\textgreen{e\_} and joint with the name of the orbital. The suffix
		should be the atomic species where the orbital placed.
        \begin{verbatim}
         e_dx2_Mo      -0.34
        \end{verbatim}

    \item[\namedlabel{tag:param-hopping}{HOPPING parameters}] $real$ \\
        The tight binding hopping parameter used in the calculations.
		\subitem $case 1.)$ \ref{tag:IS_SK} \texttt{.TRUE.} \\
		In this case, \texttt{Slater-Koster} type parameter should be
		specified properly. The syntax is as follwos:
        \begin{Verbatim}[commandchars=\\\{\}]
         \textgreen{hopping-type}_\textpink{nn-class}_\textred{A}\textblue{B}
        \end{Verbatim}
		\texttt{hopping-type} will have one of following prefix: \{ $ss$, $sp$, $sd$, $pp$, 
		$pd$, $dd$ \}, and one of following suffix: \{$s$, $p$, $d$ \}, which implies
		$\sigma$-, $\pi$-, and $\delta$-type inteaction.
		\textpink{nn-class} specifies the distance class. See \ref{tag:NNCLASS} for the 
		details.
		\textred{A}\textblue{B} specifies the two atomic species (\textred{A} and \textblue{B} atoms) 
		where the orbital hopping take place. 
		For example, for the $dd\delta$ \texttt{Slater-Koster} parameter involved with 
		the hopping process between the $\textgreen{d}_{z2}$ orbital in \textred{Mo} atom 
		and $\textgreen{d}_{yz}$ orbital in \textblue{Mo}, 
		and they are \textpink{2$^{nd}$} neighbor pair, then the parameter should be the following form:
        \begin{verbatim}
         ddd_2_MoMo   -0.2
        \end{verbatim}

		\subitem $case 2.)$ \ref{tag:IS_SK} \texttt{.FALSE.} \\
		In this case, the customized atomic orbital is assumed and the 
		following scheme should be applied:
        \begin{Verbatim}[commandchars=\\\{\}]
         \textgreen{hopping-type}_\textpink{nn-class}_\textred{A}\textblue{B}
        \end{Verbatim}
		Here, the basic structure is same as $case 1.)$, however, the syntax of 
		\textgreen{hopping-type} is slightly different. 
		That is: the prefix should have \textgreen{$cc$} since this indicates $customized$ 
		hopping parameters. For the suffix, one should put user defined 
		letter that characterize the hopping.	
		For example,
        \begin{Verbatim}[commandchars=\\\{\}]
         \textgreen{cc}\textpink{a}_2_BiBi      0.01
        \end{Verbatim}
		represents the hopping between 2$^nd$ neighbor \texttt{Bi} atoms with
		the `\textpink{$a$}` type of $rule$ which characterizes hopping pair.
		If you want to setup the $rule$, you have to write the conditions
		to the source code: \texttt{get\_cc\_param.f90}.
	

%  ! SET UP 'USER' DEFINED HOPPING RULE
%  ! NOTE: In THIS example, hopping between A-A sublattice with x-direction characterized by hopping distance
%  ! at around 11.6 ang is characterized by 'ccx_2_BiBi' since nn_class=2 is predefined in the INCAR-TB 
%  ! with Bi--Bi : 12.0 R0 12.0 where '--' gives us '2'.
%  ! Following do loop will find the parameter named 'ccx_2_BiBi' in the 'param_name' and will asign its
%  ! number as the 'parameter_index' 
	
\begin{Verbatim}[commandchars=\\\{\},gobble=4, frame=single, framesep=2mm, 
    label= source code example: get\_cc\_param.f90,
    labelposition=bottomline]

		   # SET UP THE `USER' DEFINED HOPPING `RULE'
		   # NOTE: In THIS example, hopping between Bi-Bi atom along 
		   # x-direction characterized by hopping distance at around 
		   # 8.6 \AA (cca_2_BiBi) with nn\_class = 2 will be considered.
		   # Following `\textgreen{if}' routine  will find the parameter named 
		   # cca_2_BiBi in the 'PFILE' and will asign its number as 
		   # the `parameter_index'.

	   	[...]
        \textgreen{elseif( (dij .gt.  8.5) .and. (dij .lt.  8.7)} .and. \textred{&}
					    (ci_atom .eq. cj_atom) ) then 
    	  call get_param_name(cc_custom, param_class, `\textgreen{a}', \textred{&} 
			  			     nn_class, ci_atom, cj_atom, \textred{&}
							      flag_scale)
	   	[...]

\end{Verbatim}
		
    \item[\namedlabel{tag:param-locpot}{LOCAL\_POTENTIAL}] parameters $real$ \\
		If you want to apply local potential to the particular atomic site or
		particular orbital, then you can simply turn on \ref{tag:LOCCHG}
		(\texttt{.TRUE.}) and write \texttt{local.pot} tag together with the amount of
		local potential to be applied for each atomic orbitals in the \ref{tag:GFILE}.
		Next, you have to provide proper scaling parameter ($U_{onsite}^i$) for the local 
		potential, since the local potential is applied on your Hamiltonian
		as: $e_{onsite}'^i$ = $e_{onsite}^i$ + $e_{loc.pot}^i$ $\times$ $U_{onsite}^i$,
		i.e., it modifies onsite energy $e_{onsite}^i$ to $e_{onsite}'^i$.
		Here, $U_{onsite}^i$ should be defined in your \ref{tag:PFILE} so that
		the syntax is \texttt{local\_U\_$\textgreen{orbital-type}$\_$\textred{atom-name}$}.
		$\textgreen{orbital-type}$ is one of $s$, $p$, or $d$ type of orbital and 
		$\textred{atom-name}$ is the name of atomic species you want to apply the local
		potential.
\begin{Verbatim}[commandchars=\\\{\},gobble=4, frame=single, framesep=2mm, 
    label= example of \textgreen{local.pot} tag in \ref{tag:GFILE},
    labelposition=bottomline]
  0.0 0.0 0.0 px py pz \textgreen{local.pot}  1  1  1  1  \# positive loc.pot
  0.0 0.0 0.5 px py pz \textgreen{local.pot} -1 -1 -1 -1  \# negative loc.pot

\end{Verbatim}
\begin{Verbatim}[commandchars=\\\{\},gobble=4, frame=single, framesep=2mm, 
    label= example of \textgreen{local.pot} parameter  in \ref{tag:PFILE},
    labelposition=bottomline]
    local_U_\textgreen{p}_\textred{S}  1.0
\end{Verbatim}



    \item[\namedlabel{tag:param-soc}{SOC parameters}] $real$ \\
		\subitem $case 1.)$ \ref{tag:IS_SK} \texttt{.TRUE.} \\
       	Every spin-orbit coupling parameters in \texttt{Slater-Koster} method
		should have the prefix with \texttt{lambda\_} and proper orbital information 
		\texttt{$p\_$}(as a joinder, for example $p$ orbital) 
		and species information \texttt{\_S}(as a suffix, for example 
		\texttt{Sulpur} atom) to precisely indicating the 
		atomic orbital where the SOC effect will be applied.
        \begin{verbatim}
         lambda_p_S    0.2
        \end{verbatim}

        \subitem $case 2.)$ \ref{tag:IS_SK} \texttt{.FALSE.} \\
		In the case of user defined hopping parameter (orbital prefix start with $c$,
		see Sec.\ref{tag:GFILE-detail} for the details) has been defined in the 
		\ref{tag:GFILE}, $SOC$ can be considered by setting up the Rashba and in-plane
		spin-orbit interaction.
		For Rashba type $SOC$, the prefix \texttt{lrashba\_} should be joint with 
		nearest neighbor class $n$ and hopping pair as follows.
        \begin{verbatim}
         lrashba_c_2_BiBi    0.2
        \end{verbatim}
		Above setting represents, Rashba type spin-orbit coupling between the
		custum type orbitals with $c$-prefix of the atom \texttt{Bi} and \texttt{Bi}.

	\item[\namedlabel{tag:param-fix}{Fixing parmeter}] $ $\\
		If one want some parameters not to be fitted during the fitting 
		procedures, one can fix thoes parameters by adding \texttt{FIXED} or 
		\texttt{F}.
		For example, if you want \texttt{lambda\_p\_S} to be kept as its initial 
		value, then, set this parameter as follows,
        \begin{verbatim}
         lrashba_c_2_BiBi    0.2 FIXED
        \end{verbatim}

    \item[\namedlabel{tag:param-example}{Example of \ref{tag:PFILE}}] $ $\\

\begin{Verbatim}[commandchars=\\\{\},gobble=4, frame=single, framesep=2mm, 
    label= example of PFILE: PARAM\_FIT.dat for MoS$_2$ (IS\_SK  .TRUE.),
    labelposition=bottomline]
       \textgreen{   e_dz2_Mo}      -0.34636955
       \textgreen{   e_dx2_Mo}      -0.70447045
       \textgreen{   e_dxy_Mo}      -0.70447045
       \textgreen{   e_dxz_Mo}      -0.17913534
       \textgreen{   e_dyz_Mo}      -0.17913534
       \textgreen{     e_pz_S}      -2.96500556
       \textgreen{     e_px_S}      -1.47877518
       \textgreen{     e_py_S}      -1.47877518
       \textgreen{      e_s_S}     -10.51138070
       \textgreen{ dds_1_MoMo}      -1.04598377
       \textgreen{ ddp_1_MoMo}       0.44731993
       \textgreen{ ddd_1_MoMo}       0.10237760
       \textgreen{   pps_1_SS}       0.62323972
       \textgreen{   ppp_1_SS}       0.03251328
       \textgreen{  pds_1_MoS}      -2.32384045
       \textgreen{  pdp_1_MoS}       0.97229680
       \textgreen{   sss_1_SS}      -0.57287106
       \textgreen{   sps_1_SS}      -0.33278732
       \textgreen{   sss_2_SS}      -0.45573348
       \textgreen{   sps_2_SS}      -0.21906117
       \textgreen{  sds_1_MoS}       2.66111706
       \textgreen{lambda_d_Mo}       0.08014531  Fixed
       \textgreen{ lambda_p_S}       0.07567002  Fixed

\end{Verbatim}

\begin{Verbatim}[commandchars=\\\{\},gobble=4, frame=single, framesep=2mm, 
    label= example of PFILE: PARAM\_FIT.dat for Bi/Si(110) (IS\_SK  .FALSE.),
    labelposition=bottomline]
       \textgreen{        e_cp1_Bi}    -0.09222821  
       \textgreen{      ccb_1_BiBi}     0.01723235  
       \textgreen{      cca_2_BiBi}     0.13290800  
       \textgreen{      ccy_3_BiBi}    -0.0         
       \textgreen{      ccx_4_BiBi}    -0.03544401  
       \textgreen{lrashba_c_1_BiBi}    -0.01119142  
       \textgreen{lrashba_c_2_BiBi}     0.04914549  
       \textgreen{lrashba_c_3_BiBi}    -0.00632175  
       \textgreen{lrashba_c_4_BiBi}    -0.00636364  

\end{Verbatim}
	

\end{description}

\end{document}
